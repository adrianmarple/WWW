\documentclass[10pt]{article}
\usepackage{amsmath,amsthm,amsfonts}
\usepackage{color}

\newtheorem{theorem}{Theorem}
\newtheorem{conjecture}{Conjecture}

\definecolor{background}{gray}{.2}
\definecolor{text}{gray}{.9}
\pagecolor{background}
\color{text}

\title{A Playful Theorem}
\author{Adrian Marple}

\begin{document}

\maketitle

I was sitting in the theory lounge at Stanford one day where I see a flier that says something to the effect of,
\begin{quote}
	Using only two 3s and standard arithmetic operators devise expressions that evaluate to the numbers 0 through 10.  If you solve this contact blahblah@stanford.edu for an interview.
\end{quote} 
The problem itself isn't all that hard if you think of the right arithmetic operators (unary operators seem to be particularly important).
However, as a mathematically inclined individual, once I had solved the initial problem my next instinct was to generalize.  Which leads to the following theorem

\begin{theorem}
	Given any number and standard mathematical operators (that is normal operators that do not contain numbers, letters, or Greek letters) you can always generate an expression that evaluates to an arbitrary integer.
	\label{thm:playful}
\end{theorem}

The first thing to do is to take inventory of what operators are at my disposal for this problem.  I initially only thought of unary functions from some subset of reals to reals ($\mathbb{R} \supseteq \mathbb{S} \rightarrow \mathbb{R}$), which are the following: $-,\ \sqrt{\cdot\ },\ !,\ \lfloor~\rfloor,\ \lceil~\rceil$.  After years of negligible progress with this version of the problem,\footnote{Note that it is impossible to get 0 using these operators without starting with 0 and it is impossible to escape from $[-2,2]$, so the problem statement instead involved starting with a single 3 and producing all non-zero integers.} I realized that I could use functional operators; namely $\nabla$ (del)\footnote{The del (or nabla) operator is normally only used in multivariable calculus, but here it will be used simply to denote the derivative of a function.} and $\circ$ (composition).

With this new toolbox, I finally have a proof.

\begin{proof}[Proof of Theorem~\ref{thm:playful}]
	There are some key functions to construct using the functional operators.  As a warm up let's take the first few derivatives of the square root function:
	\begin{align*}
		\nabla \sqrt{\cdot\ } (x) &= \frac{x^{-1/2}}{2}\\
		\nabla \nabla \sqrt{\cdot\ } (x) &= \frac{-x^{-3/2}}{4}\\
		\nabla \nabla \nabla \sqrt{\cdot\ } (x) &= \frac{3x^{-5/2}}{8}
	\end{align*}
	Composing these cleverly, gives the following three key functions:
	\begin{align*}
		f(x) = \nabla \sqrt{\cdot\ } \circ - \nabla \nabla \sqrt{\cdot\ }(x) &= x^{3/4}\\
		g(x) = \nabla \sqrt{\cdot\ } \circ \nabla \nabla \nabla \sqrt{\cdot\ }(x) &=  x^{5/4} \sqrt{\frac{2}{3}}\\
		h(x) = -\nabla \nabla \sqrt{\cdot\ } \circ \nabla \sqrt{\cdot\ }(x) &= x^{3/4}\sqrt{2}
	\end{align*}
	
	Let $x$ be any number.
	Notice that we can produce $\{0,1,2\}$ without too much difficulty.  If $x$ is negative apply $-$ to make it positive.  If $0 \le x < 1$,  $\lfloor x \rfloor!$ gives us $1$.  If $x > 1$, square root can be applied until the resulting number lies between $1$ and $2$, at which point applying floor yields $1$.  So far this implies that given any number we can produce $1$, and all that needs to be done is produce $0$ and $2$ from $1$.
	Because $\nabla\sqrt{\cdot\ }(1) = \frac{1}{2}$, $0$ can be produced by $\lfloor \nabla\sqrt{\cdot\ }(1) \rfloor$.  Similarly, because $h(1) = \sqrt{2}$, $2$ can be produced by $\lceil h(1) \rceil$.
	
	Now it will be argued that any positive integer can be produced by induction: given an integer $n > 1$, it is sufficient to show that $n+1$ can be produced.
	
	Consider applying $g$ to $n$ $j$ times followed by applying $f$ $i$ times.  This expression has value
		$$n^{\left(\frac{3}{4}\right)^i \left(\frac{5}{4}\right)^j} \prod_{k = 1}^{j} \left(\frac{2}{3} \right)^{2^{-2(i+k)+1}}$$
	The goal is to choose $i$ and $j$ such that the above expression is less than 1 away from $n+1$.  Taking $\log$s it is sufficient that for any $\epsilon_1 > 0$ we can find $i$ and $j$ such that
	\begin{equation}
		\left| \log(n) \left(\frac{3}{4}\right)^i \left(\frac{5}{4}\right)^j + \log\left(\frac{2}{3}\right) \sum_{k = 1}^{j} 2^{-2(i+k)+1} - \log(n+1) \right| < \epsilon_1
	\end{equation}
	Now assume we can choose $i$ arbitrarily high, this allows us to choose a new $\epsilon_2 > 0$ such that (1) is true if
	\begin{equation}
		\left| \left(\frac{3}{4}\right)^i \left(\frac{5}{4}\right)^j - \frac{\log(n+1)}{\log(n)} \right| < \epsilon_2
	\end{equation}
	By the fact that $\frac{\log(n+1)}{\log(n)}$ is positive we can again take the $\log$ of both sides and for some $\epsilon_3 > 0$, we will have (2) is true if
	\begin{equation}
		\left| i\log\left(3/4\right) +  j\log\left(5/4\right) - \log\left(\frac{\log(n+1)}{\log(n)}\right) \right| < \epsilon_3
	\end{equation}
	With some manipulation and another judicious choice of $\epsilon_4 > 0$ this yields that (3) is true if
	\begin{equation}
		\left| \frac{j}{i} - \frac{\log\left(\frac{\log(n+1)}{\log(n)}\right) - \log\left(\frac{3}{4}\right)}{\log\left(5/4\right)} \right| < \epsilon_4
	\end{equation}
	But $j$ and $i$ can be chosen to be any arbitrary positive rational number with $i$ arbitrarily high, and the term on the right is positive by basic properties of logarithms, therefore $i$ and $j$ can be chosen to satisfy (4).
	
	Altogether this implies that given any number any non-negative integer can be produced.  A simple application of minus ($-$) then produces any negative integer, completing the proof.
\end{proof}

This of course still leaves open the original version of the problem, which I leave as an open problem.

\section*{Acknowledgments}

Thanks to Jay Park for the helpful discussion.

\end{document}